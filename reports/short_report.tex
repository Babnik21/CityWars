\documentclass[a4paper, 16pt]{article}
\usepackage[slovene]{babel}
\usepackage[utf8]{inputenc}
\usepackage[T1]{fontenc}
\usepackage{lmodern}
\usepackage{hyperref}
\usepackage{graphicx}

\title{City Wars \\ Kratko poročilo}
\date{April 2022}
\author{Jure Babnik}


\begin{document}

\maketitle

\section{O igri in pravila igre}

Igra je zasnovana na podoben način kot brskalniške igre \textit{Travian}, \textit{Ikariam} in \textit{Vojna plemen}, nekaj navdiha pa jemlje tudi iz drugih strateških iger.
Igralec v igri razvija svoje mesto tako, da zgradi nove ali izboljša že obstoječe zgradbe, poleg tega pa lahko izuri vojsko. Za to potrebuje surovine, ki jih lahko pridobiva s pomočjo določenih zgradb, 
ali pa jih ukrade iz nasprotnikovih mest. S svojo vojsko lahko napade druga mesta in jih tudi osvoji. Za zmago je potrebno nadzorovati več kot polovico vseh mest na svetu.

Igra temelji na potezah; vsi igralci poteze opravijo istočasno, izvedejo pa se ko se vsi igralci odločijo za svojo potezo. V vsaki potezi ima igralec v vsakem mestu na voljo 
do največ \textbf{tri} različne akcije, ki vključujejo gradnjo/nadgradnjo zgradb, urjenje vojske ali napade na druga mesta. Na voljo je 15 različnih zgradb, kjer vsaka od njih 
opravlja določeno funkcijo in prinese določene koristi, ter 5 različnih enot, ki imajo različne kvalitete (napadalna/obrambna moč, hitrost, nosilnost plena).

Trenutno je na voljo le način za enega igralca, z 0-9 računalniškimi nasprotniki (ki trenutno še ne počnejo nič). Na svetu se pojavijo tudi samostojna mesta, ki se neodvisno razvijajo
po svoje, njihov namen pa je le ta, da jih ostali igralci zavzamejo brez večjih težav.

\section{Jezik in knjižnice}

Odločil sem se za uporabo programskega jezika \textit{Python}, saj mi je najbolj domač, igra pa ne zahteva nobenih naprednih grafičnih elementov in je knjižnica 
\textit{Pygame} več kot dovolj uporabna za moje potrebe. 

\section{Trenutno stanje igre}

Sama igra je že bolj ali manj končana, manjkajo le še malenkosti in pa estetski popravki ter kakšen bug-fix. Definiral sem nekaj razredov in napisal nekaj funkcij,
nato pa sestavil preprost vmesnik s pomočjo knjižnice \textit{Pygame}. Vmesnik ima v igri štiri različne poglede:
\begin{itemize}
    \item Pogled mesta, kjer igralec lahko vidi svoje mesto in upravlja z njim
    \item Zemljevid, kjer igralec lahko vidi druga mesta
    \item Poročila, kjer igralec lahko prebere poročila preteklih bitk
    \item Pregled, kjer igralec lahko vidi vsa svoja mesta in izbere tistega, s katerim želi upravljati
\end{itemize}
Vsak izmed njih je bolj ali manj končan, sestavljen iz preprostih oblik in sličic, ki sem jih našel na internetu. Mogoče je tudi shraniti trenutno igro in jo potem nadaljevati
ob naslednjem igranju. 

Najbolj zanimiv del projekta so računalniški nasprotniki. Do sedaj sem implementiral le navodila za razvoj neodvisnih mest. Želel sem da se obnašajo malce "neumno" in 
da se razvijajo počasneje kot ostali igralci, kar mi je uspelo na preprost način; v vsaki potezi program poišče vse mogoče poteze, nato pa po naključju izbere eno izmed potez. 
 

\section{Nadaljni načrt}

Poleg lepotnih popravkov in podobnega mi je ostal še najtežji del naloge - razviti samostojnega igralca oz. "bot"-a. Želim, da ovrednoti stanje igre le s pomočjo informacij, 
ki so na voljo tudi uporabniku in na podlagi tega izbere najboljšo možno potezo. Na ta način bi lahko preprosto implementiral tudi različne težavnosti (kjer imajo pri večji 
težavnosti nasprotniki večjo verjetnost, da izberejo optimalno potezo, pri nižji pa manjšo).

Ideja je, da napišem funkcijo, ki ovrednoti trenutno stanje igre z upoštevanjem sledečih faktorjev:
\begin{itemize}
    \item Trenutne zgradbe v vasi
    \item Trenutna vojska (v vasi in na poti)
    \item Trenutne surovine na voljo
    \item Doprinos posamezne poteze
    \item Zadnja informacija o nasprotnikovi vojski oz. ugibanje
    \item Morda še drugi
\end{itemize}

S tem bo moja igra kompletna in pripravljena za igranje.

\end{document}