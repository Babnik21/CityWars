\documentclass[a4paper, 16pt]{article}
\usepackage[slovene]{babel}
\usepackage[utf8]{inputenc}
\usepackage[T1]{fontenc}
\usepackage{lmodern}
\usepackage{hyperref}
\usepackage{graphicx}

\title{City Wars \\ Kratko poročilo}
\date{April 2022}
\author{Jure Babnik}


\begin{document}

\maketitle

\section{O igri in pravila igre}

Igra je zasnovana na podoben način kot brskalniške igre \textit{Travian}, \textit{Ikariam} in \textit{Vojna plemen}, nekaj navdiha pa jemlje tudi iz drugih strateških iger.
Igralec v igri razvija svoje mesto tako, da zgradi nove ali izboljša že obstoječe zgradbe, poleg tega pa lahko izuri vojsko. Za to potrebuje surovine, ki jih lahko pridobiva s pomočjo določenih zgradb, 
ali pa jih ukrade iz nasprotnikovih mest. S svojo vojsko lahko napade druga mesta in jih tudi osvoji. Za zmago je potrebno nadzorovati več kot polovico vseh mest na svetu.

Za razliko od zgoraj omenjenih iger, ki se igrajo v realnem času, City Wars temelji na potezah; vsi igralci poteze opravijo istočasno, izvedejo pa se ko se vsi igralci odločijo za svojo potezo. 
V vsaki potezi ima igralec v vsakem mestu na voljo 
do največ \textbf{tri} različne akcije, ki vključujejo gradnjo/nadgradnjo zgradb, urjenje vojske ali napade na druga mesta. Na voljo je 15 različnih zgradb, kjer vsaka od njih 
opravlja določeno funkcijo in prinese določene koristi, ter 5 različnih enot, ki imajo različne kvalitete (napadalna/obrambna moč, hitrost, nosilnost plena).

\subsection{Surovine}

V igri so tri različne surovine; hrana, železo in zlato. Vse tri se uporabljajo pri gradnji in urjenju. Poleg tega vojska potrebuje hrano za preživetje. Vsaka enota porabi eno enoto hrane na potezo.
Hrane in železa je na voljo veliko več, kot zlata (zlato služi bolj kot denar, in ne kot surovina/material).

\subseciton{Poteze in akcije}

Vsak igralec ima na voljo 3 akcije vsako potezo (v primeru, da ima pod nadzorom več mest, lahko začne 3 akcije v vsakem mestu). Na voljo so 4 tipi akcij:

\begin{itemize}
    \item Gradnja - igralec na prazno mesto postavi novo zgradbo
    \item Nadgradnja - igralec lahko nadgradi (izboljša) že obstoječo zgradbo, do največ stopnje 5
    \item Urjenje vojske - igralec lahko usposobi dodatne enote (če so v mestu na voljo primerne zgradbe)
    \item Napad - igralec lahko pošlje enote v napad nad drugo mesto
\end{itemize}

Vsaka akcija se prične po zaključku poteze. Gradnja in nadgradnja se izvedeta v eni potezi, kar pomeni, da so novonastale oz. nadgrajene zgradbe na voljo že v naslednji potezi.
Napadi in urjenja pa imajo različen čas trajanja. Čas trajanja (izražen v številu potez) urjenja enot je odvisen od stopnje zgradbe, v kateri se enote urijo, ne pa od količine enot. Prikazan je v spodnji tabeli.
Trajanje napadov pa je odvisno od oddaljenosti destinacije ter hitrosti najpočasnejše enote, ki sodeluje v napadu.

\begin{table}[]
    \begin{tabular}{ll|lllll}
    Zgradba      & Enota             & St. 1 & St. 2 & St. 3 & St. 4 & St. 5 \\ \hline
    Pehota       & Barake            & 3     & 3     & 2     & 2     & 1     \\
    Ostrostrelec & Strelišče         & 3     & 3     & 2     & 2     & 1     \\
    Tank         & Tovarna           & 10    & 6     & 4     & 2     & 1     \\
    Vohun        & Agencija          & 10    & 6     & 4     & 2     & 1     \\
    General      & Vojaška akademija & 10    & 6     & 4     & 2     & 1    
    \end{tabular}
\end{table}

\subsection{Zgradbe}

Mesta se pojavljajo v treh velikostih, ki omejujejo največje število zgradb. Najmanjša mesta dovoljujejo 6 zgradb, srednje velika 9, in največja 12.
Ob začetku igre vsak igralec začne v vasi z 12 polji za grajenje. 

Igralci lahko gradijo poljubne zgradbe na poljubnih mestih, edina izjema je zid, ki ima posebej določeno mesto. Ni ga mogoče zgraditi na katerokoli drugo mesto, prav tako pa ni nobene 
druge zgradbe mogoče zgraditi na mesto, ki je namenjeno zidu. Zgraditi je mogoče le eno zgradbo posameznega tipa. Ker je v igri več različnih zgradb kot je gradbenih polj (tudi v največjih mestih), 
eno izmed strateških ugank predstavlja tudi izbira zgradb, saj v igri ni mogoče rušenje. Pred gradnjo je zato treba svojo odločitev dobro premistliti.

Vsako zgradbo je mogoče nadgraditi do stopnje 5. Nadgradnja je dražja z vsako stopnjo, vendar pa višji nivo zgradbe prinaša določene koristi.

Vsaka izmed zgradb ima svoje prednosti:
\begin{itemize}
    \item Kmetija (\textit{Farm}) - proizvodnja hrane 
    \item Rudnik železa (\textit{Iron Mine}) - Proizvodnja železa
    \item Rudnik zlata (\textit{Gold Mine}) - Proizvodnja zlata
    \item Skladišče (\textit{Warehouse}) - Poveča največjo dovoljeno količino surovin
    \item Zid (\textit{Wall}) - Obrambni bonus
    \item Barake (\textit{Training Camp}) - Urjenje pehote   
    \item Bivališča (\textit{Housing}) - Poveča največjo dovoljeno količino vojske
    \item Bunker (\textit{Bunker}) - Skrije surovine pred roparji
    \item Strelišče (\textit{Range}) - Urjenje ostrostrelcev
    \item Banka (\textit{Bank}) - Poveča največjo dovoljeno količino zlata
    \item Trezor (\textit{Vault}) - Skrije zlato pred roparji
    \item Pekarna (\textit{Bakery}) - Dodatna proizvodnja hrane
    \item Agencija (\textit{Agency}) - Urjenje vohunov
    \item Tovarna (\textit{Factory}) - Izdelava tankov
    \item Vojaška akademija (\textit{Military HQ}) - Urjenje generalov
\end{itemize}

\subsection{Vojska}

V igri je na voljo 5 različnih enot. Vsaka enota ima določeno ceno urjenja oz. izdelave. Poleg tega ima vsaka izmed enot naslednje atribute:
\begin{itemize}
    \item Napadalna moč
    \item Obrambna moč
    \item Hitrost premikanja (izražena v poljih na potezo, ki jih lahko prehodi)
    \item Nosilnost (koliko surovin lahko nosijo)
\end{itemize}

Spodnja tabela prikazuje vse atribute enot in ceno njihovega urjenja oz. izdelave.

\begin{table}[]
    \begin{tabular}{l|llll|lll}
    Enota        & Napadalna moč & Obrambna moč & Hitrost premikanja & Nosilnost & Hrana & Železo & Zlato \\ \hline
    Pehota       & 50            & 25           & 4                  & 100       & 35    & 15     & 0     \\
    Ostrostrelec & 10            & 70           & 3                  & 75        & 30    & 25     & 0     \\
    Tank         & 150           & 190          & 1                  & 200       & 150   & 200    & 5     \\
    Vohun        & 1             & 10           & 7                  & 0         & 100   & 10     & 3     \\
    General      & 0             & 0            & 1                  & 0         & 1000  & 100    & 100  
    \end{tabular}
\end{table}

Vohun se uporablja za ocenjevanje nasprotnikove vojaške moči, saj lahko opazijo sovražnikove enote, brez da bi nasprotnik opazil njih.
Posebno funkcijo pa ima tudi general, ki v primeru vojaške zmage zavzame nasprotnikovo mesto in igralcu omogoči nadzor nad njim.

\subsection{Napadi}

Igralec ima na voljo 4 različne tipe napadov:

\begin{itemize}
    \item Polni napad (\textit{Attack}) - enote napadejo z namenom, da pobijejo nasprotnikovo vojsko
    \item Roparski napad (\textit{Raid}) - enote napadejo z namenom, da ukradejo surovine
    \item Vohunjenje (\textit{Espionage}) - enote napadejo z namenom, da odkrijejo nasprotnikove vojaške sposobnosti (količino vojske)
    \item Prevzem (\textit{Conquest}) - enote napadejo z namenom, da pobijejo nasprotnikovo vojsko in prevzamejo nadzor nad mestom
\end{itemize}

Po vsakem napadu obe mesti prejmeta poročilo, v katerem so napisani podatki o napadalni in obrambni vojski, skupaj z izgubami obeh strani.
V polnih napadih in prevzemih se vojaki pobijejo, dokler ne ostanejo le še pripadniki ene strani. Pri roparskih napadih se pobijejo v manjšem številu, izgube pa so sorazmerne z 
razmerjem moči napadalcev in branilcev. V kolikor katere napadalčeve enote preživijo, ukradejo tudi nekaj surovin in jih prinesejo v mesto, od koder prihajajo.
Na enak način se enote spopadajo tudi pri vohunjenju, vendar pa v primeru, ko obrambno mesto nima prisotnega nobenega lastnega vohuna, do spopada sploh ne pride (v tem primeru branilec sploh ne prejme poročila o bitki).

Za vohunjenje so primerni le vohuni; zato je vohunjenje mogoče izvesti le, če so v poslani vojski prisotni samo vohuni, za 
prevzem pa more vojska vsebovati vsaj enega generala.

\section{Jezik in knjižnice}

Odločil sem se za uporabo programskega jezika \textit{Python}, saj mi je najbolj domač, igra pa ne zahteva nobenih naprednih grafičnih elementov in je knjižnica 
\textit{Pygame} več kot dovolj uporabna za moje potrebe. 

\section{Trenutno stanje igre}

Sama igra je že bolj ali manj končana, manjkajo le še malenkosti in pa estetski popravki ter kakšen bug-fix. Definiral sem nekaj razredov in napisal nekaj funkcij,
nato pa sestavil preprost vmesnik s pomočjo knjižnice \textit{Pygame}. Vmesnik ima v igri štiri različne poglede:
\begin{itemize}
    \item Pogled mesta, kjer igralec lahko vidi svoje mesto in upravlja z njim
    \item Zemljevid, kjer igralec lahko vidi druga mesta
    \item Poročila, kjer igralec lahko prebere poročila preteklih bitk
    \item Pregled, kjer igralec lahko vidi vsa svoja mesta in izbere tistega, s katerim želi upravljati
\end{itemize}
Vsak izmed njih je bolj ali manj končan, sestavljen iz preprostih oblik in sličic, ki sem jih našel na internetu. Mogoče je tudi shraniti trenutno igro in jo potem nadaljevati
ob naslednjem igranju. 

Najbolj zanimiv del projekta so računalniški nasprotniki. Do sedaj sem implementiral le navodila za razvoj neodvisnih mest. Želel sem da se obnašajo malce "neumno" in 
da se razvijajo počasneje kot ostali igralci, kar mi je uspelo na preprost način; v vsaki potezi program poišče vse mogoče poteze, nato pa po naključju izbere eno izmed potez. 
 

\section{Nadaljni načrt}

Poleg lepotnih popravkov in podobnega mi je ostal še najtežji del naloge - razviti samostojnega igralca oz. "bot"-a. Želim, da ovrednoti stanje igre le s pomočjo informacij, 
ki so na voljo tudi uporabniku in na podlagi tega izbere najboljšo možno potezo. Na ta način bi lahko preprosto implementiral tudi različne težavnosti (kjer imajo pri večji 
težavnosti nasprotniki večjo verjetnost, da izberejo optimalno potezo, pri nižji pa manjšo).

Ideja je, da napišem funkcijo, ki ovrednoti trenutno stanje igre z upoštevanjem sledečih faktorjev:
\begin{itemize}
    \item Trenutne zgradbe v vasi
    \item Trenutna vojska (v vasi in na poti)
    \item Trenutne surovine na voljo
    \item Doprinos posamezne poteze
    \item Zadnja informacija o nasprotnikovi vojski oz. ugibanje
    \item Morda še drugi
\end{itemize}

S tem bo moja igra kompletna in pripravljena za igranje.

\end{document}